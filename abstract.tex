Memory analysis is an established technique for malware analysis and is
increasingly used for incident response. However, in most incident response
situations, the responder often has no control over the precise version of the
operating system that must be responded to. It is therefore critical to ensure
that memory analysis tools are able to work with a wide range of OS kernel
versions, as found in the wild. This paper characterizes the properties of
different Windows kernel versions and their relevance to memory analysis. By
collecting a large number of kernel binaries we characterize how struct offsets
change with versions. We find that although struct layout is mostly stable
across major and minor kernel versions, kernel global offsets vary greatly with
version. We develop a ``profile indexing'' technique to rapidly detect the exact
kernel version present in a memory image. We can therefore directly use known
kernel global offsets and do not need to guess those by scanning techniques. We
demonstrate that struct offsets can be rapidly deduced from analysis of kernel
pool allocations, as well as by automatic disassembly of binary functions. As an
example of an undocumented kernel driver, we use the {\em win32k.sys} GUI
subsystem driver and develop a robust technique for combining both profile
constants and reversed struct offsets into accurate profiles, detected using a
profile index.
