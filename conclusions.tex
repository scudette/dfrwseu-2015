Although this paper concentrates specifically on the windows kernel binary and
the win32k.sys GUI subsystem driver, the techniques presented are applicable for
other drivers and binaries.

Specifically the tcpip.sys driver manages the network stack and is largely
undocumented. The same techniques we develop for constructing profiles from a
mixture of documented and undocumented (reversed) information can be applied to
this case.

Identifying which of a set of known binaries matches the exact running binary in
a memory image is a critical first step to memory analysis of all operating
systems. For example, we have extended this method to auto-detect the exact
kernel running on an OSX system.

The ability to generate profiles with more accurate information allows one to
abandon using scanning and guessing techniques for determining this information
from the potentially compromised memory image itself. The less the framework
relies on the memory image to derive analysis information, the more resilient it
is to malicious manipulation. For example, the literature has noted that the
Kernel Debugger Block can be easily overwritten by malware in such a way that
memory analysis can fail to find it \cite{haruyama2012one}.

Finally this paper presents the groundwork for ultimately addressing the
difficult problem of Linux memory analysis. Linux kernel struct layouts vary
wildely based on kernel configuration as well as purely on kernel version. Only
recently has it become possible to acquire memory on a Linux system in a kernel
version agnostic manner \citep{stuttgen2014robust}, but there is a wide need to
reliably determine the correct profile for unknown kernels - often encountered
during incident response situations.

Previously, systems were proposed that attempted to derive all kernel struct
offsets by examining the specific assembly instructions. However these systems,
failed to take into account register swapping and function rebranching
\cite{case2010dynamic}, making them less reliable for matching real kernels in
practice. This paper's proposed assember templates are much more robust to these
variations. Furthermore, previous dynamic analysis platoforms attempt to build a
complete profile from the reversed parameters. However, as shown in this paper,
we only need to gather just enough information to select the correct profile
from a finite set of known profile variations. Future work can apply the
techniques discussed in this paper to auto-detecting a Linux profile from an
unknown kernel.
