This study's main goal was to characterize what factors change between various
binary versions, and how these are relevant to memory analysis. We found that
generally struct layout does not change within the same minor version, but
global constants were found found to vary wildly with version.


In our quest to characterize the veriation we have developed a number of very
useful techniques:

\begin{enumerate}
\item We have developed a technique to build a ``profile index'' - a mechanism
  to quickly detect which profile from a pre-calculated profile repository is
  applicable for a specific memory image. Our method is resilient to
  anti-forensic manipulation since it uses a random selection of comparison
  points chosen from the binary code and data segments themselves.

\item We have also demonstrated a data analysis technique for rapidly
  determining struct offsets by analyzing kernel pool allocations.

\item We have created an Assembler templating language which can be used to
  match sequences of assembler code in order to extract struct offsets for
  struct members. This technique can be applied for static binaries as well as
  binaries found in memory images.
\end{enumerate}

How should these techniques be applied in order to improve the accuracy of
memory analysis software?

As noted previously, some memory analysis frameworks currently use techniques
such as pool scanning, disassembling and other heuristics to guess the locations
of global kernel variables \cite{volatility}. This is especially problematic
when trying to locate {\em win32k.sys} global parameters, since the GUI
subsystem has a different pool area for each session. Without contextual
information, pool scanning techniques can not associate the correct kernel
structures to the correct session, leading to many erroneous results.

It is desirable therefore, to rely on accurate profile information in locating
global structures. This warrants the creation and maintenance of a public
profile repository with accurate symbol information for each version observed in
the wild \citep{rekall-profiles}. The problem remains however, how does one know
which profile should be used for a specific memory image?

By applying the profile indexing technique, one can reliably detect the correct
profile to use for each memory image. The profiles can then contain exact
offsets of global variables and functions. This improves analysis because there
is a large amount of accurate information available (for example it is possible
to resolve addresses to function names - really helping with disassembly views).

Finally we can address the problem of undocumented struct layouts. While the
win32k.sys profiles do contain the addresses of global variables and functions,
most do not contain struct layout.

Although we can apply the assembler templates to deduce the struct layouts in a
memory image, this is not a reliable technique, since in practice many code
pages will not be mapped into memory - causing the disassembly of the required
functions to fail.

Instead we can collect win32k.sys binaries of all major and minor versions and
apply the disassembly templates to the binaries themselves. Although we can
never be absolutely sure that struct layouts are the same in all builds of the
same version, our analysis suggests this is the case. That is, the struct layout
for win32k.sys depend only on the major and minor version numbers of the
win32k.sys binary itself.

Therefore we can construct a profile for all win32k.sys binaries by merging the
global constants and functions found in the PDB files provided by Microsoft with
the canonical struct layout for the specific major and minor version. We then
similarly create a ``profile index'' for all known win32k.sys profiles, and
apply it on in the memory image to detect the correct profile to use.

Once the correct profile is found (containing both accurate constants and
accurate struct layouts) we can use it to conduct analysis of the memory image
without problems.


