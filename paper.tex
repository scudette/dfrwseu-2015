%%
%% Copyright 2007, 2008, 2009 Elsevier Ltd
%%
%% This file is part of the 'Elsarticle Bundle'.
%% ---------------------------------------------
%%
%% It may be distributed under the conditions of the LaTeX Project Public
%% License, either version 1.2 of this license or (at your option) any
%% later version.  The latest version of this license is in
%%    http://www.latex-project.org/lppl.txt
%% and version 1.2 or later is part of all distributions of LaTeX
%% version 1999/12/01 or later.
%%
%% The list of all files belonging to the 'Elsarticle Bundle' is
%% given in the file `manifest.txt'.
%%
%% Template article for Elsevier's document class `elsarticle'
%% with harvard style bibliographic references
%% SP 2008/03/01

%\documentclass[preprint,12pt,5p,authoryear]{elsarticle}

%% Use the option review to obtain double line spacing
%% \documentclass[authoryear,preprint,review,12pt]{elsarticle}

%% Use the options 1p,twocolumn; 3p; 3p,twocolumn; 5p; or 5p,twocolumn
%% for a journal layout:
%% \documentclass[final,1p,times,authoryear]{elsarticle}
%% \documentclass[final,1p,times,twocolumn,authoryear]{elsarticle}
%% \documentclass[final,3p,times,authoryear]{elsarticle}
%% \documentclass[final,3p,times,twocolumn,authoryear]{elsarticle}
%% \documentclass[final,5p,times,authoryear]{elsarticle}
\documentclass[final,5p,times,twocolumn,authoryear,10pt]{elsarticle}

%% For including figures, graphicx.sty has been loaded in
%% elsarticle.cls. If you prefer to use the old commands
%% please give \usepackage{epsfig}

%% The amssymb package provides various useful mathematical symbols
\usepackage{amssymb}
\usepackage{subfigure}
\usepackage{listings}
\usepackage{soul}
\usepackage{xcolor}

\lstnewenvironment{assembly}
{\lstset{
    language=[x86masm]Assembler,
    basicstyle=\tiny\ttfamily,
    keywordstyle=\color{blue},
    commentstyle=\color[rgb]{0.13,0.54,0.13},
    morekeywords={CDQE,CQO,CMPSQ,CMPXCHG16B,JRCXZ,LODSQ,MOVSXD,
      POPFQ,PUSHFQ,SCASQ,STOSQ,IRETQ,RDTSCP,SWAPGS,
      RAX,RDX,RCX,RBX,RSI,RDI,RSP,RBP,
      R8,R8D,R8W,R8B,R9,R9D,R9W,R9B,
      PUSH, SUB, TEST, JMP, MOV, JZ, CALL, POP, RET, CMP, ADD, JNZ,
    },
    frame=single,
    linewidth=\columnwidth,
    escapechar=@,
}}
{}

%% The amsthm package provides extended theorem environments
%% \usepackage{amsthm}

%% The lineno packages adds line numbers. Start line numbering with
%% \begin{linenumbers}, end it with \end{linenumbers}. Or switch it on
%% for the whole article with \linenumbers.
%% \usepackage{lineno}

\journal{Digital Investigation}

\begin{document}

\begin{frontmatter}

%% Title, authors and addresses

%% use the tnoteref command within \title for footnotes;
%% use the tnotetext command for theassociated footnote;
%% use the fnref command within \author or \address for footnotes;
%% use the fntext command for theassociated footnote;
%% use the corref command within \author for corresponding author footnotes;
%% use the cortext command for theassociated footnote;
%% use the ead command for the email address,
%% and the form \ead[url] for the home page:
%% \title{Title\tnoteref{label1}}
%% \tnotetext[label1]{}
%% \author{Name\corref{cor1}\fnref{label2}}
%% \ead{email address}
%% \ead[url]{home page}
%% \fntext[label2]{}
%% \cortext[cor1]{}
%% \address{Address\fnref{label3}}
%% \fntext[label3]{}

\title{Characterization of the Windows Kernel version variability for accurate Memory analysis.}

%% use optional labels to link authors explicitly to addresses:
%% \author[google]{Michael Cohen}
%% \address[google]{Google Inc., Brandschenkestrasse 110, Zurich, Switzerland}

\author[google]{Redacted for Blind Review}
\address[google]{Redacted for Blind Review}

\begin{abstract}
Memory analysis is an established technique for malware analysis and is
increasingly used for incident response. However, in most incident response
situations, the responder often has no control over the precise version of the
operating system that must be responded to. It is therefore critical to ensure
that memory analysis tools are able to work with a wide range of OS kernel
versions, as found in the wild. This paper characterizes the properties of
different Windows kernel versions and their relevance to memory analysis. By
collecting a large number of kernel binaries we characterize how struct offsets
change with versions. We find that although struct layout is mostly stable
across major and minor kernel versions, kernel global offsets vary greatly with
version. We develop a ``profile indexing'' technique to rapidly detect the exact
kernel version present in a memory image. We can therefore directly use known
kernel global offsets and do not need to guess those by scanning techniques. We
demonstrate that struct offsets can be rapidly deduced from analysis of kernel
pool allocations, as well as by automatic disassembly of binary functions. As an
example of an undocumented kernel driver, we use the {\em win32k.sys} GUI
subsystem driver and develop a robust technique for combining both profile
constants and reversed struct offsets into accurate profiles, detected using a
profile index.

\end{abstract}

\begin{keyword}
Memory analysis \sep Incident response \sep Binary classification \sep Memory Forensics
%% keywords here, in the form: keyword \sep keyword

%% PACS codes here, in the form: \PACS code \sep code

%% MSC codes here, in the form: \MSC code \sep code
%% or \MSC[2008] code \sep code (2000 is the default)

\end{keyword}

\end{frontmatter}

%% \linenumbers

%% main text
\section{Introduction}
Memory analysis has become a powerful technique for the detection and
identification of malware, and for digital forensic investigations
\citep{ligh2010malware,Ligh:2014:AMF:2621971}.

Fundamentally. Memory analysis is concerned with interpreting the seemingly
unstructured raw memory data which can be collected from a live system into
meaningful and actionable information. At first sight, the memory content of a
live system might appear to be composed of nothing more than random
bytes. However, those bytes are arranged in a predetermined order by the running
software to represent a meaningful data structure. For example consider the C
struct:

\begin{lstlisting}
typedef struct _EPROCESS {
  unsigned long long CreateTime;
  char[16] ImageFileName;
} EPROCESS;

\end{lstlisting}


The compiler will decide how to overlay the struct fields in memory depending on
their size, alignment requirements and other consideration. So for example, the
{\em CreateTime} field might get 8 bytes, causing the {\em ImageFileName} field to
begin 8 bytes after the start of the {\em \_EPROCESS} struct.

A memory analysis framework must have the same layout information in order to
know where each field should be found in relation to the start of the
struct. Early memory analysis systems hard coded this layout information which
was derived by other means (e.g. reverse engineering or simply counting the
fields in the struct header file. \citep{schuster2007ptfinder}).

This approach is not scalable though, since the struct definition change
routinely between versions of the operating system. For example, in the above
simplified struct of an {\em \_EPROCESS}, if additional fields are inserted, the
layout of the field members will change to make room for the new elements. So
for example, if another 4 byte field is added before the {\em CreateTime} field
- all other offsets will have to increase by 4 bytes to accommodate the new
field. This will cause all the old layout information to be incorrect and our
interpretation of the struct in memory to be wrong.

Modern memory analysis frameworks address the variations across different
operating system versions by use of a version specific memory layout template
mechanism. For example in Volatility \citep{volatility} or Rekall \citep{rekall}
this information is called a {\em profile}.

The Volatility memory analysis framework \citep{volatility} is shipped with a
number of Windows profiles embedded into the program. The user chooses the
correct profile to use depending on their image. For example, if analyzing a
Windows 7 image, the profile might be specified as {\em Win7SP1x64}. In
Volatility, the profile name conveys major version information (i.e. Windows 7),
minor version information (i.e. Service Pack 1) and architecture
(i.e. x64). Volatility uses this information to select a profile from the set of
built-in profiles.

\subsection{Deriving profile information}
The problem still remains how to derive this struct layout information
automatically. The windows kernel contains many struct definitions, and these
change for each version, so a brute force solution is not scalable
\citep{okolica2010windows}.

Memory analysis frameworks are not the only case where information about memory
layout is required. Specifically, when debugging an application, the debugger
needs to know how to interpret the memory of the debugged program in order to
correctly display it to the user. Since the compiler is the one originally
deciding on the memory layout, it makes sense that the compiler generates
debugging information about memory layout for the debugger to use.

On Windows systems, the most common compiler used is the Microsoft Visual Studio
compiler (MSVCC). This compiler shares debugging information via a {\em PDB}
file \citep{Schreiber:2001:UWS:375734}, generated during the build process for
the executable. The PDB file format is unfortunately undocumented, but has been
reverse engineered sufficiently to be able to extract accurate debugging
information, such as struct memory layout, reliably
\citep{Schreiber:2001:UWS:375734,moyix2007}.

The PDB file for an executable is normally not shipped together with the
executable. The executable contains a unique GUID referring to the PDB file that
describes this executable. When the debugger wishes to debug a particular
executable, it can then request the correct PDB file from a {\em symbol
  server}. This design allows production binaries to be debugged, without
needing to ship bulky debug information with final release binaries.

The PDB file contains a number of useful pieces of information for a memory
analysis framework:
\begin{itemize}

\item Struct members and memory layout. This contains information about memory
  offsets for struct members, and their types. This is useful in order to
  interpret the contents of memory.

\item Global constants. The Windows kernel contains many important constants,
  which are required for analysis. For example, the {\em PsActiveProcessHead} is
  a constant pointer to the beginning of the process linked list, and is
  required in order to list processes by walking that list.

\item Function addresses. The location of functions in memory is also provided
  in the PDB file - even if these functions are not exported. This is important
  in order to resolve addresses back to functions (e.g. in viewing the IDT).

\item Enumeration. In C an enumeration is a compact way to represent one of a
  set of choices using an integer. The mapping between the integer value and a
  human meaningful string is stored in the PDB file, and it is useful for
  interpreting meaning from memory.
\end{itemize}

\section{Characterizing Kernel version variability}
\label{kernel_variability}
As described previously, the Volatility tool only contains a handful of profiles
generated for different major releases of the windows kernel. However, each time
the kernel is rebuilt by Microsoft (e.g. for a security hot fix), the code could
be changed, and the profile could be different. The assumption made by the
Volatility tool is that these changes are not significant and therefore, a
profile generated from a single version of a major release will work on all
versions from that release.

We wanted to validate this assumption. We collected the windows kernel binary
({\em ntkrnlmp.exe, ntkrpamp.exe, ntoskrnl.exe}) from several thousand machines
in the wild using the GRR tool \citep{cohen2011distributed}. Each of these
binaries has a unique GUID, and therefore we downloaded the corresponding PDB
file from the public Microsoft symbol server. We then used Rekall's {\em mspdb}
parser to extract debugging information from each PDB file.

This resulted in 168 different binaries of the windows kernel for various
versions (e.g. Windows XP, Windows Vista, Windows 7 and Windows 8) and
architectures (e.g. I386 and AMD64). Clearly there are many more versions of the
windows kernel in the wild than exist in the Volatility tool. It is also very
likely that we have not collected all the versions that were ever released by
Microsoft, so our sample size, although large, is not exhaustive.

\begin{figure}[tb]
\begin{center}
\includegraphics[width=1\columnwidth]{results/structs_AMD64}
\includegraphics[width=1\columnwidth]{results/structs_I386}
\caption{Offsets for a few critical struct members across various versions of
  the windows kernel. These offsets were derived by analyzing public debug
  information from the Microsoft debug server for the binaries in our
  collection.
}
\label{graph_structs}
\end{center}
\end{figure}

Figure \ref{graph_structs} shows sampled offsets of four critical struct members
for memory analysis:
\begin{itemize}
\item The {\em \_EPROCESS.VadRoot} is the location of the Vad within the
  process. This is used to enumerate process allocations \citep{dolan2007vad}.

\item The {\em \_KPROCESS.DirectoryTableBase} is the location of the Directory
  Table Base (i.e. the value loaded into CR3) which is critical in constructing
  the Virtual Address Space abstraction.

\item The {\em \_EPROCESS.ImageFileName} is the file name of the running
  binary. For example, this field might contain ``csrss.exe''.
\end{itemize}

Microsoft windows kernel versions contain four parts: The major and minor
versions, the revision and the build number. The build number increases for each
build (e.g. security hotfix).

As can be seen in the figure, struct offsets do tend to remain stable across
windows versions. In most cases, with a single notable exception - version
5.2.3970.175 (GUID 466B4165EAA84AF88D29D617E86A95982), the struct offsets remain
the same for all major windows releases. Therefore, chances are good that the
Volatility profile for a give windows version would actually work most of the
time for determining struct layout.

\subsection{Kernel global constants variability}
It is generally not sufficient to determine only the struct memory layout for
memory analysis. For example, consider listing the running processes. One
technique is to follow the doubly linked list of {\em
  EPROCESS.ActiveProcessLinks} in each process struct
\citep{okolica2010windows}. However, we generally need to find the start of the
list which begins at the global kernel constant {\em PsActiveProcessHead}. The
location for this global constant in memory is determined statically by the
compiler at compile time, and it is usually stored in one of the data sections
in the PE file itself.

Since this information is also required by the debugger, the PDB file also
contains information about global constants and functions (even if these are not
actually exported via the Export Address Table). Rekall's {\em mspdb} plugin
also extract this information into the profile.

\begin{figure}[tb]
\begin{center}
\includegraphics[width=1\columnwidth]{results/constants_AMD64}
\includegraphics[width=1\columnwidth]{results/constants_I386}
\caption{Offsets for a few global kernel constants across various versions of
  the windows kernel. These offsets were derived by analyzing public debug
  information from the Microsoft debug server for the binaries in our
  collection. Offsets are provided relative to the kernel image base address.
}
\label{graph_constants}
\end{center}
\end{figure}

Figure \ref{graph_constants} illustrate the memory addresses of some important
kernel constants for the kernels in our collection:

\begin{itemize}
\item {\em NtBuildLab} is the location of the NT version string
  (e.g. ``7600.win7\_rtm.090713-1255''). This is used to identify the running
  kernel.

\item {\em PsActiveProcessHead} is the head of the active process list. This is
  required in order to list the running processes.

\item {\em NtCreateToken} is an example of a kernel function. This will normally
  exist in the {\em .text} section of the PE file.

\item {\em str:FILE\_VERSION} is literally the string ``FILE\_VERSION''. Usually
  the compiler will place all literal strings into their own string table in the
  {\em .rdata} section of the PE file. The compiler will then emit debugging
  symbols for the location of each string - indicating that they are literal
  strings. The importance of this symbol will be discussed in the following
  sections.

\end{itemize}

As can be seen, the offsets of global kernel constants change dramatically
between each build - even for the same version. This makes sense, since the
compiler arranges global constants in their own PE section, so if any global
constant is added or removed in the entire kernel, this affects the ordering of
all other constants placed after it.

It is therefore clear that it is unreliable to directly obtain the addresses of
kernel globals by simply relying on the version alone. The Volatility tool
resorts to a number of techniques to obtain these globals:

\begin{itemize}
\item Many globals are obtained from the {\em KdDebuggerDataBlock} - another
  global kernel struct which contains pointers to many other globals (This is
  usually scanned for).

\item Scanning for kernel objects which refer to global constants (e.g. via pool
  tag scanning or other signatures.).

\item Examining the export tables of various PE binaries for exported functions.

\item Dynamically disassembling code to detect calls to non exported functions.
\end{itemize}

These techniques are complex and error prone. They are also susceptible to
anti-forensics as signature scanners can trivially be fooled by spurious
signatures \citep{add}. Scanning for signatures over very large memory images is
also slow and inefficient.

The Rekall memory forensic framework \citep{rekall}, a fork of the Volatility
framework, takes a different approach. Instead of guessing the location of
various kernel constants, the framework relies on a public profile repository
which contains every known profile from every known build of the windows
kernel. This greatly simplifies memory analysis algorithms because the address
of global kernel variables and functions is directly known from public debugging
information provided by Microsoft; There is no need to scan or guess at
all. Locating these globals is very efficient since there is no need to scan for
signatures, making the framework fast and reducing the ability of attackers to
subvert analysis.

\section{Identifying binary versions}
\label{identifying}

The Rekall profile repository contains, at the time of writing, 309 profiles for
various windows kernel versions (and this number is constantly
increasing). Typically, users will simply report the GUID of the windows kernel
found in their image, but will not provide the actual kernel binary.

Previously, Rekall employed a scanning technique to locate the GUID of the NT
kernel running within the image. Once the GUID is known, the correct profile can
be fetched from the repository and analysis can begin. However this technique is
still susceptible to manipulation (It is easy for attackers to simply wipe or
alter the GUID from memory). Sometimes the GUID is paged out of memory and in
this case it is impossible to guess it. What we really need is a reliable way to
identify the kernel version without relying on a single signature.

The problem of identifying kernel binaries in a memory image has been examined
previously in the Linux memory analysis context \citep{roussev2014image}. In
that paper, the authors used similarity hashing to match the kernel in a memory
image with a corpus of known binaries.

In our case, we do not always have the actual binaries but have debugging
symbols from these binaries. We therefore need a way for deducing enough
information about the kernel binary itself (which we may not have) from the
debug symbols. Consider the following information present in the PDB file:

\begin{itemize}
\item String Literals. As shown in the example above, the compiler generates
  string literals in the PE binary itself. These are then located using global
  debugging symbols. For example, in Figure \ref{graph_constants} we know the
  exact offsets in memory where we expect find the string ``FILE\_VERSION''.

\item Function preamble. The PDB file also contains the locations of many
  functions. We note that each function is generally preceded by 5 NOP
  instructions in order to make room for hot patching \citep{hot_patching}. Thus
  we can deduce that for each function in the PDB, the previous byte contains
  the value 0x90 (NOP instruction).
\end{itemize}

The problem, therefore, boils down to identifying which of a finite set of
kernel profiles is the one present in the memory image, based on known data that
must exist at known offsets:

\begin{enumerate}
\item Begin by selecting a number of function names, or literal string names. We
  term these {\em Comparison Points} since we only compare the binaries at these
  known offsets.

\item Examine all available profiles, and record the offset of these symbols,
  and the expected data to appear at this offset (either a NOP instruction or
  the literal string itself).

\item Build a decision tree around the known comparison points to minimize the
  number of string comparisons required for narrowing down the match. Note that
  at this stage it is possible to determine if there are sufficient number of
  comparison points to distinguish all profile selections. If profile selection
  is ambiguous, further comparison points are added and the process starts
  again.

\item Scan the memory image for the longest strings using the Aho-Corasick
  string matching algorithm \citep{aho1975efficient}.

\item For each match, seek around the match to apply the decision tree
  calculated earlier. Within a few string comparisons, the correct profile is
  identified.

\item Load the profile from the profile repository and initialize the analysis.
\end{enumerate}

In practice it was found that fewer than a dozen comparison points are required
to characterize all the profiles in the Rekall profile repository, leading to
extremely quick matching times. Binary identification is robust to manipulation
since the choice of comparison points is rather arbitrary and can be changed
easily.

\subsection{Windows kernel binary identification}
Section \ref{identifying} described an efficient algorithm for identifying a
binary match from a set of known binaries. However, in the memory analysis
context, this comparison must be made in the Virtual address space. Modern CPUs
operate in protected mode, and the exact memory accessible to the kernel does
not necessarily need to be contiguous in the physical memory image.

Therefore, before we are able to apply the index classification algorithm, we
must build a virtual address space, requiring us to identify the value of CR3,
or the kernel's Directory Table Base (DTB).

The DTB can be captured during the acquisition process and stored in the image,
but typically it must be scanned for. The Volatility memory forensic framework
scans for the Idle process's {\em EPROCESS} struct. It first searches for the
literal string ``Idle'', this should exist as the {\em EPROCESS.ImageFileName}
member. Knowing the difference between the offsets of {\em
  EPROCESS.ImageFileName} and {\em EPROCESS.Pcb.DirectoryTableBase}, the
framework reads the DTB and therefore locates the page tables.

The problem with this approach is that it requires knowing the exact offsets of
two {\em EPROCESS} struct members. Figure \ref{graph_structs} shows how these
relative offsets vary between windows versions, so to know the offset we need to
know the exact windows version we are examining - but we can not identify the
profile without applying the profile index, which requires a valid kernel
address space - i.e. knowing the DTB first!

We solve this Catch-22 by noting that the total number of combinations of the
{\em EPROCESS} member offsets is limited (4 combinations for 64 bit
architectures and 6 combinations for 32 bit architectures). Therefore it is
possible to brute force all combinations in search of a valid DTB.

So in summary the complete Kernel Binary Autodetection algorithm is:

\begin{itemize}
\item Scan the image for common windows executable names (e.g. ``csrss.exe'',
  ``cmd.exe'' etc). This scan uses the Aho-Corasick algorithm to search for all
  strings at once.

\item For each hit, brute force the DTB going through the 10 possible
  offsets. The DTB is validated using the {\em
    KUSER\_SHARED\_DATA.NtMajorVersion} and {\em
    KUSER\_SHARED\_DATA.NtMinorVersion} members. Since this struct must be found
  at a fixed location in memory and always have the same layout it is safe to
  hardcode it \citep{temporal}. Therefore we can validate the DTB and kernel
  address space without knowing anything about the profile itself or the kernel
  version.

\item Once a DTB is identified, we construct a virtual address space, and scan
  for the kernel image in memory using the algorithm previously described.

\end{itemize}

\section{Undocumented kernel structures.}
\label{undocumented}

Section \ref{kernel_variability} examined the variability of documented kernel
structures across different kernel versions. However, what is the variability of
undocumented kernel structures of significance to the memory analyst?

One of the most interesting kernel drivers is the Windows 32 user mode GUI
subsystem \citep{mandt2011kernel,yuan2001windows}, implemented as
``win32k.sys''. The data structures used in this subsystem are required to
detect many common hooks placed by malware (e.g. {\em SetWindowsHookEx()} style
keyloggers \citep{Sikorski:2012:PMA:2181153}).

The Rekall profile repository currently contains profiles for 169 unique
versions of this driver. However, only 33 versions include information about
critical structures (e.g. {\em tagDESKTOP} and {\em tagWINDOWSTATION}). The
remaining profiles only contain information about global constants and
functions, but no structure information.

Our goal is to understand how various important structures evolved through the
released versions. Since many of these versions are not documented and do not
have debugging information, previous researches have manually reverse engineered
several samples from different versions. However, we are unsure if there is
internal variability within windows versions and releases. Guided by our
previous experience with the Windows Kernel versions, we hypothesize that the
{\em win32k.sys} struct layout would not vary much between minor release
versions.

Given our large corpus of binaries we can directly examine this hypothesis and
evaluate the best approach for determining struct layout when analyzing the
Win32k GUI subsystem.

\subsection{Data driven reverse engineering.}
\label{data_driven_reversing}

The literature contains a number of published systems for automatically
detecting kernel objects from memory images \citep{sun2012reversing}. For
example, the SigGraph system \citep{lin2011siggraph}, is capable of building
scanners for Linux kernel structures by analyzing their internal pointer graphs.

The SigGraph system, specifically does not utilize incidental knowledge about
the system to assist in the reversing task. However on Windows systems, there
are some helpful observation one can make to make type analysis from memory
dumps easier.

In the Windows kernel all allocations come from one of the kernel pools
(e.g. Paged, Non-Paged or Session Pool). Allocations smaller than a page are
preceded by a {\em POOL\_HEADER} object
\citep{schuster2006pool,schuster2008impact}.

\begin{figure}[tb]
\begin{center}
\includegraphics[width=0.7\columnwidth]{diagrams/structs}
\caption{An example of a typical Windows Kernel pool allocation. The {\em
    POOL\_HEADER} indicates the type of the allocation. This header is also part
  of a doubly linked list with the next/previous allocation - a relation which
  may be used to validate it. By observing the type of allocations the struct
  members are pointing to it is possible to deduce the pointers.
}
\label{pool_allocations}
\end{center}
\end{figure}


The pool header contains a known tag as well as indications of the previous and
next pool allocation (within the page). Thus small pool allocations form a
doubly linked list. Due to this property it is possible to validate the pool
header and therefore locate it in memory.

If we were to ask, ``What kernel object exists at a given virtual offset?'', we
simply scan backwards for a suitable {\em POOL\_HEADER} structure, and from the
pool tag, we can deduce the type of object. We can further scan forward from
this location for other heuristics, such as pointers to certain other pool
allocations, or doubly linked lists. We wrote a Rekall plugin called {\em
  analyze\_structs} to perform this analysis on arbitrary memory locations.

For example, Figure \ref{analyze_structs} shows the analysis of the global
symbol {\em grpWinStaList} which is the global offset of the head of the {\em
  tagWINDOWSTATION} list.  We can see that at offset 0x10 there is a pointer to
the {\em tagDESKTOP} object, at offset 0x18 there is a pointer to the global
gTermIO object etc.

With Windows 7 we can find the complete struct information in the PDB file.
This is also shown in Figure \ref{analyze_structs}. We can see that the detected
pointers correspond with the {\em rpdeskList, pTerm, spklList, pGlobalAtomTable}
and {\em psidUser} members.

An obvious limitation of this technique is that if a pointer in the struct is
set to NULL, we are unable to say anything about it. Hence to reveal as many
fields as possible we need to examine as many instances of the same object type
as we can find (e.g. via pool scanning techniques).

\begin{figure}
\label{analyze_structs}
\begin{lstlisting}[basicstyle=\tiny\ttfamily, frame=single, linewidth=\columnwidth]
win7.elf 22:39:38> analyze_struct ``*win32k!grpWinStaList''
0xfa80022b0090 is inside pool allocation with tag 'Win\xe4' (0xfa80022b0000)
    Offset     Content
-------------- -------
          0x10 Data:0xfa8001853bc0 Tag:Des\xeb @0xfa8001853bc0
          0x18 Data:0xf960003af340 Const:win32k!gTermIO
          0x28 Data:0xf900c01369f0 Tag:Uskb @0xf900c01369f0
          0x78 Data:0xf8a002953880 Tag:AtmT @0xf8a002953880
          0x90 Data:0xf900c1aa0880 Tag:Usse ProcessBilled:winlogon.exe

win7.elf 22:42:14> print win32k_profile.tagWINDOWSTATION(0xFA80022B0090)
[tagWINDOWSTATION tagWINDOWSTATION] @ 0xFA80022B0090
  0x00 dwSessionId             [unsigned long:dwSessionId]: 0x00000001
  0x08 rpwinstaNext           <tagWINDOWSTATION Pointer to [0x00000000]>
  0x10 rpdeskList             <tagDESKTOP Pointer to [0xFA8001853BC0]>
  0x18 pTerm                  <tagTERMINAL Pointer to [0xF960003AF340]>
  0x20 dwWSF_Flags             [unsigned long:dwWSF_Flags]: 0x00000000
  0x28 spklList               <tagKL Pointer to [0xF900C01369F0]>
  0x30 ptiClipLock            <tagTHREADINFO Pointer to [0x00000000]>
  0x38 ptiDrawingClipboard    <tagTHREADINFO Pointer to [0x00000000]>
  0x40 spwndClipOpen          <tagWND Pointer to [0x00000000]>
  0x48 spwndClipViewer        <tagWND Pointer to [0x00000000]>
  0x50 spwndClipOwner         <tagWND Pointer to [0x00000000]>
  0x58 pClipBase              <Array Pointer to [0x00000000]>
  0x60 cNumClipFormats         [unsigned long:cNumClipFormats]: 0x00000000
  0x64 iClipSerialNumber       [unsigned long:iClipSerialNumber]: 0x00000000
  0x68 iClipSequenceNumber     [unsigned long:iClipSequenceNumber]: 0x00000000
  0x70 spwndClipboardListener <tagWND Pointer to [0x00000000]>
  0x78 pGlobalAtomTable       <_RTL_ATOM_TABLE Pointer to [0xF8A002953880]>
  0x80 luidEndSession         [_LUID luidEndSession] @ 0xFA80022B0110
  0x88 luidUser               [_LUID luidUser] @ 0xFA80022B0118
  0x90 psidUser               <Void Pointer to [0xF900C1AA0880]>

\end{lstlisting}
\caption{Rekall analysis of the global symbol {\em grpWinStaList} which contains
  an allocation of type {\em tagWINDOWSTATION}. This is followed by the exact
  struct layout as extracted from the PDB file.  }
\end{figure}

\subsection{Code based reverse engineering.}
The previous section demonstrates how we can deduce some struct layouts by
observation of allocations we can find from the kernel pools. However, these
observations are not sufficient to deduce all types of members. Specifically,
only pointers are reliably deduced by this method. Additionally we must observe
allocated memory in a memory dump from a running system. Often we only have the
executable binary (e.g. from disk) but not the full memory image.

In these cases, we need to resort to the more traditional reverse engineering
approach.  Previous researches have reversed specific exemplars of the win32k
dll which is representative of major windows version
\citep{volatility}. However, manually reverse engineering every file in our
large corpus of {\em win32k.sys} binaries is time consuming and error
prone. Additionally, some tools simply contain the reversed profile data as
``Magic Numbers'' \citep{volatility} without an explanation of where these numbers
came from, making forensic validation and cross checking difficult.

We wish to automatically extend this analysis to new binaries with minimal
effort. We therefore wish to express the required assembler pattern as a
template which can be applied to the new file's disassembly. In practice,
however, the compiler is free to mix use of registers in functions, or reorder
branches. Often identical source code will generate assembler code using
different registers, and different branching order.

Figure \ref{disassembly} shows the same code segment from two different versions
of the {\em xxxCreateWindowStation} function. As can be seen, although the
general sequence of instructions is similar, the exact registers are different
for each case (This function essentially checks the {\em rpdesk} pointer of the
global variable {\em gptiCurrent}, a global {\em tagTHREADINFO} struct). We
therefore construct our patten match in such as way that exact register names
are not specified, but we require the same register to be used for {\em \$var1}
throughout the pattern. Our template can now be published and independently
cross validated for accuracy. For example, in the event that an investigator
finds a different version of the binary in the wild, they are able to apply the
templates and re-derive the struct offsets directly from the binary - cross
validating the resulting profile.

\begin{figure}
\label{disassembly}
\begin{assembly}
 MOV RSI, [RIP+0x276b62]        0x0 win32k!gptiCurrent
 MOV RBP, [RSI+0x178]
 TEST RBP, RBP
\end{assembly}

\begin{assembly}
 MOV RSI, [RIP+0x2bbfdc]        0x0 win32k!gptiCurrent
 MOV R15, [RSI+0x190]
 TEST R15, R15
\end{assembly}

\begin{lstlisting}[basicstyle=\tiny\ttfamily, frame=single, linewidth=\columnwidth]
tagTHREADINFO:
  rpdesk:
  - - Disassembler
    - rules:
      - MOV $var1, *gptiCurrent
      - MOV $var2, [$var1+$out]
      - TEST $var2, $var2
      start: win32k!xxxCreateWindowStation
      target: Pointer
      target_args:
        target: tagDESKTOP

\end{lstlisting}
\caption{Disassembled code for finding the {\em tagTHREADINFO.rpdesk} member
  offset. Even though the code is identical, different versions use different
  registers. We define a search template (Below) in YAML format to describe the
  required pattern regardless of the exact registers used.  }
\end{figure}

Additionally, the compiler may reorder Assembler code fragments from version to
version. When a branch is reordered, the pattern match may be split into
different parts of the branching instruction. In order to normalize the effect
of branching, we unroll all branches in the assembly output. This means we
follow all branches until we reach code that is already disassembled, and then
backtrack to resume disassembly from the branch onwards. This technique allows
us to match our pattern against the complete code of each function.

\begin{figure}[tb]
\label{disassembly2}
\begin{assembly}
Address                     Instruction        Comment
------------------ ------------------------- -------------------------------
------ win32k!SetGlobalCursorLevel ------: 0xf97fff1dc0a4
 0xf97fff1dc0a4    PUSH RBX
 0xf97fff1dc0a6    SUB RSP, 0x20
 0xf97fff1dc0aa    @\hl{MOV RAX, [RIP+0x1b362f]   win32k!grpdeskRitInput}@
 0xf97fff1dc0b1    MOV EBX, ECX
 0xf97fff1dc0b3    @\hl{TEST RAX, RAX}@
 0xf97fff1dc0b6    JZ 0xf97fff1dc0da         win32k!SetGlobalCursorLevel+0x36
. 0xf97fff1dc0da   MOV RCX, [RIP+0x1b3827]   win32k!gpepCSRSS
. 0xf97fff1dc0e1   CALL QWORD [RIP+0x151ce9] win32k!imp_PsGetProcessWin32Process
. 0xf97fff1dc0e7   MOV RCX, [RAX+0x128]      ; tagPROCESSINFO.ptiList
. 0xf97fff1dc0ee   JMP 0xf97fff1dc11f        win32k!SetGlobalCursorLevel+0x7b
.. 0xf97fff1dc11f  TEST RCX, RCX
.. 0xf97fff1dc122  JNZ 0xf97fff1dc105        win32k!SetGlobalCursorLevel+0x61
... 0xf97fff1dc105 MOV RAX, [RCX+0x178]      ; tagTHREADINFO.pq
... 0xf97fff1dc10c MOV [RCX+0x290], EBX      ; tagTHREADINFO.iCursorLevel
... 0xf97fff1dc112 MOV [RAX+0x148], EBX      ; tagQ.iCursorLevel
... 0xf97fff1dc118 MOV RCX, [RCX+0x238]      ; tagTHREADINFO.ptiSibling
.. 0xf97fff1dc124  ADD RSP, 0x20
.. 0xf97fff1dc128  POP RBX
.. 0xf97fff1dc129  RET
 0xf97fff1dc0b8    @\hl{MOV RAX, [RAX+0x18]}@          ; tagDESKTOP.rpwinstaParent
 0xf97fff1dc0bc    @\hl{MOV RDX, [RAX+0x10]}@          ; tagWINDOWSTATION.rpdeskList
 0xf97fff1dc0c0    JMP 0xf97fff1dc0d5        win32k!SetGlobalCursorLevel+0x31
. 0xf97fff1dc0d5   TEST RDX, RDX
. 0xf97fff1dc0d8   JNZ 0xf97fff1dc0c2        win32k!SetGlobalCursorLevel+0x1e
.. 0xf97fff1dc0c2  @\hl{LEA R8, [RDX+0xa0]}@           ; tagDESKTOP.PtiList
.. 0xf97fff1dc0c9  MOV RCX, [R8]             ; _LIST_ENTRY.Flink
.. 0xf97fff1dc0cc  CMP RCX, R8
.. 0xf97fff1dc0cf  JNZ 0xf97fff1dc0f0        win32k!SetGlobalCursorLevel+0x4c
... 0xf97fff1dc0f0 MOV RAX, [RCX-0x108]      ; tagTHREADINFO.PtiLink
... 0xf97fff1dc0f7 MOV [RCX+0x10], EBX
... 0xf97fff1dc0fa MOV [RAX+0x148], EBX      ; tagQ.iCursorLevel
... 0xf97fff1dc100 MOV RCX, [RCX]
... 0xf97fff1dc103 JMP 0xf97fff1dc0cc        win32k!SetGlobalCursorLevel+0x28
.. 0xf97fff1dc0d1  MOV RDX, [RDX+0x10]
\end{assembly}

\begin{lstlisting}[basicstyle=\tiny\ttfamily, frame=single, linewidth=\columnwidth]
tagDESKTOP:
  PtiList:
  - - Disassembler
    - rules:
      - MOV $var1, *grpdeskRitInput
      - TEST $var1, $var1
      - MOV $var1, [$var1+$rpwinstaParent]
      - MOV $pdesk, [$var1+$rpdeskList]
      - LEA *, [$pdesk+$out]
      start: win32k!SetGlobalCursorLevel
      target: Pointer
      target_args:
        target: _LIST_ENTRY
      max_separation: 300
\end{lstlisting}

\caption{An example of matching an assembler pattern across a short
  function. First the function is unrolled such that all its branches are
  displayed. The pattern is then applied such that the same registers are used
  in a consistent manner. By comparing the assembly code to the struct field
  offsets in the exported PDB we can easily infer the types of structs used in
  this function. We can then extrapolate this inference to deduce struct offsets
  for binary versions we have no debugging information for.
}
\end{figure}

For example consider Figure \ref{disassembly2}. This shows a very short function
{\em win32k!SetGlobalCursorLevel} which dereferences many pointers to a number
of structs. The function iterates over all desktops ({\em tagDESKTOP}) and all
threads ({\em tagTHREADINFO}) and sets their cursor level. It is quite simple to
infer the structs and fields involved when reading the assembly code (for
Windows 7) in conjunction with the struct definitions exported in the PDB files
for Windows 7. The same templates can then be applied for other versions of the
binary for which there are no exported symbols.

It must be noted that this technique does not work in every case since the code
does change from version to version, sometimes dramatically. We therefore offer
a number of possibile templates (to different functions) that can be applied in
turn until a match is found.


\label{}


\section{Results}
We have collected 133 unique versions of the ``win32k.sys'' driver binary, and
downloaded PDB files for these samples. We then generated assembler templates
for many struct fields and ran these templates over these binaries in our
collections.

\begin{figure}[tb]
\begin{center}
\includegraphics[width=1\columnwidth]{results/win32k_AMD64_structs}
\includegraphics[width=1\columnwidth]{results/win32k_I386_structs}
\caption{Offsets for a selection of struct members across various versions of
  the windows GUI subsystem. These offsets were derived by applying the
  automated disassembly templates on the driver executable.
}
\label{win32k_structs}
\end{center}
\end{figure}

Figure \ref{win32k_structs} shows a summary of struct offsets across different
versions of the win32k driver. As can be seen the struct offsets are generally
not changed between major and minor binary versions, although they do vary
between each minor version.


\begin{figure}[tb]
\begin{center}
\includegraphics[width=1\columnwidth]{results/win32k_AMD64_constants}
\includegraphics[width=1\columnwidth]{results/win32k_I386_constants}
\caption{Offsets for a selection of global constants across various versions of
  the windows GUI subsystem. These offsets were derived by parsing the provided
  PDB files for these binary versions.
}
\label{win32k_constants}
\end{center}
\end{figure}

Simiarly Figure \ref{win32k_constants} shows that global constants vary wildly
from build to build, hence version number alone is insufficient to provide
reliable offsets for these constants.




\section{Discussion}
This study's main goal was to characterize what factors change between various
binary versions, and how these are relevant to memory analysis. We found that
generally struct layout does not change within the same minor version, but
global constants were found found to vary wildly with version.


In our quest to characterize the veriation we have developed a number of very
useful techniques:

\begin{enumerate}
\item We have developed a technique to build a ``profile index'' - a mechanism
  to quickly detect which profile from a pre-calculated profile repository is
  applicable for a specific memory image. Our method is resilient to
  anti-forensic manipulation since it uses a random selection of comparison
  points chosen from the binary code and data segments themselves.

\item We have also demonstrated a data analysis technique for rapidly
  determining struct offsets by analyzing kernel pool allocations.

\item We have created an Assembler templating language which can be used to
  match sequences of assembler code in order to extract struct offsets for
  struct members. This technique can be applied for static binaries as well as
  binaries found in memory images.
\end{enumerate}

How should these techniques be applied in order to improve the accuracy of
memory analysis software?

As noted previously, some memory analysis frameworks currently use techniques
such as pool scanning, disassembling and other heuristics to guess the locations
of global kernel variables \cite{volatility}. This is especially problematic
when trying to locate {\em win32k.sys} global parameters, since the GUI
subsystem has a different pool area for each session. Without contextual
information, pool scanning techniques can not associate the correct kernel
structures to the correct session, leading to many erroneous results.

It is desirable therefore, to rely on accurate profile information in locating
global structures. This warrants the creation and maintenance of a public
profile repository with accurate symbol information for each version observed in
the wild \citep{rekall-profiles}. The problem remains however, how does one know
which profile should be used for a specific memory image?

By applying the profile indexing technique, one can reliably detect the correct
profile to use for each memory image. The profiles can then contain exact
offsets of global variables and functions. This improves analysis because there
is a large amount of accurate information available (for example it is possible
to resolve addresses to function names - really helping with disassembly views).

Finally we can address the problem of undocumented struct layouts. While the
win32k.sys profiles do contain the addresses of global variables and functions,
most do not contain struct layout.

Although we can apply the assembler templates to deduce the struct layouts in a
memory image, this is not a reliable technique, since in practice many code
pages will not be mapped into memory - causing the disassembly of the required
functions to fail.

Instead we can collect win32k.sys binaries of all major and minor versions and
apply the disassembly templates to the binaries themselves. Although we can
never be absolutely sure that struct layouts are the same in all builds of the
same version, our analysis suggests this is the case. That is, the struct layout
for win32k.sys depend only on the major and minor version numbers of the
win32k.sys binary itself.

Therefore we can construct a profile for all win32k.sys binaries by merging the
global constants and functions found in the PDB files provided by Microsoft with
the canonical struct layout for the specific major and minor version. We then
similarly create a ``profile index'' for all known win32k.sys profiles, and
apply it on in the memory image to detect the correct profile to use.

Once the correct profile is found (containing both accurate constants and
accurate struct layouts) we can use it to conduct analysis of the memory image
without problems.




\section{Conclusions and future work}
Although this paper concentrates specifically on the windows kernel binary and
the win32k.sys GUI subsystem driver, the techniques presented are applicable for
other drivers and binaries.

Specifically the tcpip.sys driver manages the network stack and is largely
undocumented. The same techniques we develop for constructing profiles from a
mixture of documented and undocumented (reversed) information can be applied to
this case.

Identifying which of a set of known binaries matches the exact running binary in
a memory image is a critical first step to memory analysis of all operating
systems. For example, we have extended this method to auto-detect the exact
kernel running on an OSX system.

The ability to generate profiles with more accurate information allows one to
abandon using scanning and guessing techniques for determining this information
from the potentially compromised memory image itself. The less the framework
relies on the memory image to derive analysis information, the more resilient it
is to malicious manipulation. For example, the literature has noted that the
Kernel Debugger Block can be easily overwritten by malware in such a way that
memory analysis can fail to find it \cite{haruyama2012one}.

Finally this paper presents the groundwork for ultimately addressing the
difficult problem of Linux memory analysis. Linux kernel struct layouts vary
wildely based on kernel configuration as well as purely on kernel version. Only
recently has it become possible to acquire memory on a Linux system in a kernel
version agnostic manner \citep{stuttgen2014robust}, but there is a wide need to
reliably determine the correct profile for unknown kernels - often encountered
during incident response situations.

Previously, systems were proposed that attempted to derive all kernel struct
offsets by examining the specific assembly instructions. However these systems,
failed to take into account register swapping and function rebranching
\cite{case2010dynamic}, making them less reliable for matching real kernels in
practice. This paper's proposed assember templates are much more robust to these
variations. Furthermore, previous dynamic analysis platoforms attempt to build a
complete profile from the reversed parameters. However, as shown in this paper,
we only need to gather just enough information to select the correct profile
from a finite set of known profile variations. Future work can apply the
techniques discussed in this paper to auto-detecting a Linux profile from an
unknown kernel.


%% The Appendices part is started with the command \appendix;
%% appendix sections are then done as normal sections
%% \appendix

%% \section{}
%% \label{}

%% If you have bibdatabase file and want bibtex to generate the
%% bibitems, please use
%%
\section{References}
\bibliographystyle{elsarticle-harv}
\bibliography{refs}

\end{document}

\endinput
%%
%% End of file `elsarticle-template-harv.tex'.
